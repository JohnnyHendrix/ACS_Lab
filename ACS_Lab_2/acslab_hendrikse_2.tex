\documentclass{semdoc}
\usepackage{hyperref}
\usepackage[utf8]{inputenc}
\usepackage{listings}
\usepackage{color}
\bibliographystyle{alphadin}
\definecolor{codegreen}{rgb}{0,0.6,0}
\definecolor{codegray}{rgb}{0.5,0.5,0.5}
\definecolor{codepurple}{rgb}{0.58,0,0.82}
\definecolor{backcolour}{rgb}{0.95,0.95,0.92}
\lstdefinestyle{mystyle}{
   backgroundcolor=\color{backcolour},
   commentstyle=\color{codegreen},
   keywordstyle=\color{magenta},
   numberstyle=\tiny\color{codegray},
   stringstyle=\color{codepurple},
   basicstyle=\footnotesize,
   breakatwhitespace=false,
   breaklines=true,
   captionpos=b,
   keepspaces=true,
   showspaces=false,
   showstringspaces=false,
   showtabs=false,
   tabsize=2
}
\lstset{style=mystyle}
% Template: $Id: t01_txt.tex,v 1.7 2000/05/23 12:13:37 bless Exp $
% -----------------------------------------------------------------------------
%epstopdf ermöglicht, dass eps-Dateien durch pdflatex in windows eingebunden werden können
%\usepackage{epstopdf}
% Report Praktikum
% -----------------------------------------------------------------------------
% Kommentare beginnen mit einem %-Zeichen
\docbegin
% --> Oberhalb der Linie bitte nichts ändern.
% ---------------------------------------------------------------------------
% \/ \/ \/ \/ \/ \/ \/ \/ \/ \/ \/ \/ \/ \/ \/ \/ \/ \/ \/ \/ \/ \/ \/ \/ \/
% Stellen, an denen etwas geaendert werden soll, sind wie hier gekennzeichnet.
% /\ /\ /\ /\ /\ /\ /\ /\ /\ /\ /\ /\ /\ /\ /\ /\ /\ /\ /\ /\ /\ /\ /\ /\ /\

%
% ---------------------------------------------------------------------------
% \/ \/ \/ \/ \/ \/ \/ \/ \/ \/ \/ \/ \/ \/ \/ \/ \/ \/ \/ \/ \/ \/ \/ \/ \/
% --> Bitte den Titel des Beitrages in die nächste Zeile eintragen:
\title{Praktikumsbericht 2}
%
% --> ... und den Namen des Autors:
\author{Jean-Marc Hendrikse - 1751591}
% /\ /\ /\ /\ /\ /\ /\ /\ /\ /\ /\ /\ /\ /\ /\ /\ /\ /\ /\ /\ /\ /\ /\ /\ /\
% -----------------------------------------------------------------------------

% Nicht ändern!
\event{Access Control Systems Lab\\}
\term{Sommersemester 2017}
\supervisor{Prof. Dr. Hannes Hartenstein, Alexander Degitz, Jan Grashöfer, Till Neudecker}

%
%
\maketitle

\section*{Introduction}
\label{Introduction}
As mentioned in the last lab protocol Linux Security Modules (LSM) is a Framework
that allows various implementations of a Reference Monitor for Linux.
We dived into one of these implementations - the Security-Enhanced Linux (SELinux)
and used it as a Reference Monitor.
In this paper we are going to build our own LSM.
We will learn how to compile the LSM into the kernel on which we will work.
% ---------------------------------------------------------------------------
% \/ \/ \/ \/ \/ \/ \/ \/ \/ \/ \/ \/ \/ \/ \/ \/ \/ \/ \/ \/ \/ \/ \/ \/ \/
% Preparation section.
% /\ /\ /\ /\ /\ /\ /\ /\ /\ /\ /\ /\ /\ /\ /\ /\ /\ /\ /\ /\ /\ /\ /\ /\ /\

%
% ---------------------------------------------------------------------------

\section*{Preparations}
\label{preparation}
The preparation part for writing our own LSM is the most important part because during
development process it will helped you to come back to these steps and rebuild the kernel
if you change the kernel code. Because it is so important I would like to summarize
what you should do for preparation:
\begin{enumerate}
  \item First of all we installed necessary dependencies via \texttt{sudo dnf install openssl-devel elfutils-libelf-devel}
  \item Afterwards we cloned the Linux source tree via \texttt{git clone -b acslab --single-branch https://<user>@git.kit.edu/dsn-stud-projects/acs-lab/linux.git}
  \item In the next steps we followed the guide at  \url{https://fedoraproject.org/wiki/Building_a_custom_kernel#Building_Vanilla_upstream_kernel}
  \item We copied the existing system configuration file \texttt{cp /boot/config-'uname -r'* .config } into our linux/ directory.
  \item After the configuration we started to build the kernel.\footnote{Every single step from above is detailed described in the task sheet of this lab
  lesson and can be read at \url{https://fedoraproject.org/wiki/Building_a_custom_kernel#Building_Vanilla_upstream_kernel}} \\
  \texttt{\$ make oldconfig}\\
  \texttt{\$ make bzImage}\\
  \texttt{\$ make modules}\\
  (become root)\\
  \texttt{\# make modules\_install}\\
  \texttt{\# make install}\\
\end{enumerate}

% ---------------------------------------------------------------------------
% \/ \/ \/ \/ \/ \/ \/ \/ \/ \/ \/ \/ \/ \/ \/ \/ \/ \/ \/ \/ \/ \/ \/ \/ \/
% 1 Understanding the ACS-Lab LSM template
% /\ /\ /\ /\ /\ /\ /\ /\ /\ /\ /\ /\ /\ /\ /\ /\ /\ /\ /\ /\ /\ /\ /\ /\ /\

%
% ---------------------------------------------------------------------------
\section{Understanding the ACS-Lab LSM template}
After checking out the repository from preparation section
we get a playground for writing a LSM called ``acslab''.
To see the necessary changes between last commit and the kernel
sources before Version 4.9 to build a LSM we use the command \texttt{git diff v4.9}.
\\
\begin{lstlisting}[language=bash, caption={git diff v4.9 output}, captionpos=b, label=gitdiff]
[student@DSN-ACSLab-Master linux]$ git diff v4.9
diff --git a/Makefile b/Makefile
index b103777..424aa0e 100644
--- a/Makefile
+++ b/Makefile
@@ -1,7 +1,7 @@
 VERSION = 4
 PATCHLEVEL = 9
 SUBLEVEL = 0
-EXTRAVERSION =
+EXTRAVERSION = jean
 NAME = Roaring Lionus

 # *DOCUMENTATION*
diff --git a/include/linux/lsm_hooks.h b/include/linux/lsm_hooks.h
index 558adfa..5bccd2f 100644
--- a/include/linux/lsm_hooks.h
+++ b/include/linux/lsm_hooks.h
@@ -1933,5 +1933,10 @@ void __init loadpin_add_hooks(void);
 #else
 static inline void loadpin_add_hooks(void) { };
 #endif
+#ifdef CONFIG_SECURITY_ACSLAB
+extern void __init acslab_add_hooks(void);
+#else
+static inline void __init acslab_add_hooks(void) { }
+#endif

 #endif /* ! __LINUX_LSM_HOOKS_H */
diff --git a/security/Kconfig b/security/Kconfig
index 118f454..656ac24 100644
--- a/security/Kconfig
+++ b/security/Kconfig
@@ -164,6 +164,7 @@ source security/tomoyo/Kconfig
 source security/apparmor/Kconfig
 source security/loadpin/Kconfig
 source security/yama/Kconfig
+source security/acslab/Kconfig

 source security/integrity/Kconfig

diff --git a/security/Makefile b/security/Makefile
index f2d71cd..6f9865b 100644
--- a/security/Makefile
+++ b/security/Makefile
@@ -9,6 +9,7 @@ subdir-$(CONFIG_SECURITY_TOMOYO)        += tomoyo
 subdir-$(CONFIG_SECURITY_APPARMOR)     += apparmor
 subdir-$(CONFIG_SECURITY_YAMA)         += yama
 subdir-$(CONFIG_SECURITY_LOADPIN)      += loadpin
+subdir-$(CONFIG_SECURITY_ACSLAB)       += acslab CONFIG_SECURITY_ACSLAB = y

 # always enable default capabilities
 obj-y                                  += commoncap.o
@@ -24,6 +25,7 @@ obj-$(CONFIG_SECURITY_TOMOYO)         += tomoyo/
 obj-$(CONFIG_SECURITY_APPARMOR)                += apparmor/
 obj-$(CONFIG_SECURITY_YAMA)            += yama/
 obj-$(CONFIG_SECURITY_LOADPIN)         += loadpin/
+obj-$(CONFIG_SECURITY_ACSLAB)          += acslab/
 obj-$(CONFIG_CGROUP_DEVICE)            += device_cgroup.o

 # Object integrity file lists
diff --git a/security/acslab/Kconfig b/security/acslab/Kconfig
new file mode 100644
index 0000000..b217e9a
--- /dev/null
+++ b/security/acslab/Kconfig
@@ -0,0 +1,6 @@
+config SECURITY_ACSLAB
+       bool "ACS-Lab LSM support"
+       depends on SECURITY_PATH
+       default n
+       help
+         This module contains an exemplary LSM for the ACS-Lab.
diff --git a/security/acslab/Makefile b/security/acslab/Makefile
new file mode 100644
index 0000000..9bcfedb
--- /dev/null
+++ b/security/acslab/Makefile
@@ -0,0 +1 @@
+obj-$(CONFIG_SECURITY_ACSLAB) += acslab.o
diff --git a/security/acslab/acslab.c b/security/acslab/acslab.c
new file mode 100644
index 0000000..821673f
--- /dev/null
+++ b/security/acslab/acslab.c
@@ -0,0 +1,72 @@
+/*
@@ -0,0 +1,72 @@
+/*
+ * Playground module for the LSM framework.
+ *
+ */
+
+#define pr_fmt(fmt) "ACS-Lab: " fmt
+
+#include <linux/lsm_hooks.h>
+#include <linux/time64.h>      // timespec64
+#include <linux/time.h>                // timezone
+#include <linux/path.h>                // path
+#include <linux/dcache.h>      // dentry_path
+#include <linux/string.h>      // strnstr
+#include <linux/usb.h>         // USB
+
+/*** Hook handler definitions ***/
+
+/*
+#define VENDOR_ID 0x0000
+#define PRODUCT_ID 0x0000
+
+static int match_usb_dev(struct usb_device *dev, void *unused)
+{
+       return ((dev->descriptor.idVendor == VENDOR_ID) &&
+               (dev->descriptor.idProduct == PRODUCT_ID));
+}
+
+static int acslab_path_mkdir(const struct path *dir, struct dentry *dentry, umode_t mode)
+{
+       char buf_dir[256];
+       char buf_path[256];
+       char *ret_dir;
+       char *ret_path;
+
+       ret_dir = dentry_path(dir->dentry, buf_dir, ARRAY_SIZE(buf_dir));
+       if (IS_ERR(ret_dir)) {
+               pr_info("mkdir hooked: <failed to retrieve directory>\n");
+               return 0;
+       }
+
+       ret_path = dentry_path(dentry, buf_path, ARRAY_SIZE(buf_path));
+       if (IS_ERR(ret_path)) {
+               pr_info("mkdir hooked: <failed to retrieve path>\n");
+               return 0;
+       }
+
+       pr_info("mkdir hooked: %s in %s\n", ret_path, ret_dir);
+
+       // Add your code here //
+
+       return 0;
+}
+*/
+
+static int acslab_settime (const struct timespec64 *ts, const struct timezone *tz)
+{
+       pr_info("settime hooked\n");
+       return 0;
+}
+
+/*** Add hook handler ***/
+
+static struct security_hook_list acslab_hooks[] = {
+       //LSM_HOOK_INIT(path_mkdir, acslab_path_mkdir),
+       LSM_HOOK_INIT(settime, acslab_settime),
+};
+
+void __init acslab_add_hooks(void)
+{
+       pr_info("Hooks have been added!");
+       security_add_hooks(acslab_hooks, ARRAY_SIZE(acslab_hooks));
+}
diff --git a/security/security.c b/security/security.c
index f825304..a99b356 100644
--- a/security/security.c
+++ b/security/security.c
@@ -61,6 +61,7 @@ int __init security_init(void)
        capability_add_hooks();
        yama_add_hooks();
        loadpin_add_hooks();
+       acslab_add_hooks();
\end{lstlisting}
\begin{itemize}
\item As you can see in lisitng \ref{gitdiff} we configured \texttt{EXTRAVERSION}
and set it to \texttt{jean}.

\item In the LSM hooks \textit{linux/lsm\_hooks.h} an if-else definition is appended
which defines that \texttt{extern void \_\_init acslab\_add\_hooks(void);} is called if
\texttt{CONFIG\_SECURITY\_ACSLAB} is set and if not we jump into the \texttt{else}
clause and execute \texttt{static inline void \_\_init acslab\_add\_hooks(void) \{ \}}

\item In the security configuration file with path \textit{security/Kconfig} a source path
of our acslab security config file was added \textit{source security/acslab/Kconfig}.

\item Subdirectory acslab was created because we set CONFIG\_SECURITY\_ACSLAB=y:
subdir-\$(CONFIG\_SECURITY\_ACSLAB)	+= acslab in \textit{security/Makefile}.

\item The newly created security config is filled with content and definitions that
 configures the Security ACS-Lab.

\item The linked object file is defined in the acslab Makefile.

\item The acslab C file for kernel implementation will be edited in the next sections.
For now it serves a framework with all necessary definitions and methods we implement later.

\item Also in security.c the acslab\_add\_hooks() was appended
\end{itemize}

\\
ACS-Lab LSM is a minor LSM. In the acslab.c file there are two functions which are
stateless what is an indicator for a minor LSM. Furthermore, the hooks do not use
security IDs (secid) like selinux a major LSM does.

%b) Explain whether the ACS-Lab LSM is a minor or major LSM
%- Requires no blobs
%- called after
	%- traditional controls
  %  - capabilities
%- called before any major module
\section{Exploring the ACS-Lab LSM}
After we compiled the kernel successfully and examined all necessary changes to
the LSM framework to configure our own LSM,
we will start to explore the ACS-Lab LSM. To see what the ACS-Lab LSM does we have to
look into the kernel logs.
Every access decisions by the LSM is logged right here. To get those messages
containing the prefix \textit{ACS-Lab} we just use the command \texttt{jorunalctl}

\begin{lstlisting}[language=bash, caption={git diff v4.9 output}, captionpos=b, label=gitdiff]
$ journalctl -k | grep 'ACS-Lab'
May 23 15:46:35 DSN-ACSLab-Master kernel: ACS-Lab: Hooks have been added!
May 23 15:46:38 DSN-ACSLab-Master kernel: ACS-Lab: settime hooked
\end{lstlisting}

Let’s change the system time to see what happens in the log messages. For changing
the system time we use the command \texttt{timedatectl} and set as parameter a
date and time, e.g. your birthday:

\begin{lstlisting}[language=bash, caption={git diff v4.9 output}, captionpos=b,
  label=gitdiff]
$ sudo timedatectl set-time '1990-09-10 18:00:00'

[student@DSN-ACSLab-Master ~]$ journalctl -k | grep "ACS-Lab"
May 23 15:46:35 DSN-ACSLab-Master kernel: ACS-Lab: Hooks have been added!
May 23 15:46:38 DSN-ACSLab-Master kernel: ACS-Lab: settime hooked
Sep 10 18:00:00 DSN-ACSLab-Master kernel: ACS-Lab: settime hooked
\end{lstlisting}

\section{Preventing the Creation of Directories}
In this section, we would like to do some fancy things with the kernel like preventing
the user to create a certain directory with name \textit{dsn}. We will only
permitt the user with a physical access key such as a `magic' USB device which
should be attached to the system. To do so we must edit the code
in \texttt{acslab.c}:

\begin{lstlisting}[numbers=left, numberstyle=\small, numbersep=8pt, xleftmargin=1.5em, framexleftmargin=1.5em,language=C++, caption={Implemented acslab\_path\_mkdir() acslab.c}, captionpos=b, label=acslabc]
  /*
   * Playground module for the LSM framework.
   *
   */

  #define pr_fmt(fmt) "ACS-Lab: " fmt

  #include <linux/lsm_hooks.h>
  #include <linux/time64.h>	// timespec64
  #include <linux/time.h>         // timezone
  #include <linux/path.h>         // path
  #include <linux/dcache.h>	// dentry_path
  #include <linux/string.h>	// strnstr
  #include <linux/usb.h>          // USB
  /*** Hook handler definitions ***/


  #define VENDOR_ID 0x090c
  #define PRODUCT_ID 0x1000

  static int match_usb_dev(struct usb_device *dev, void *unused)
  {
          return ((dev->descriptor.idVendor == VENDOR_ID) &&
                  (dev->descriptor.idProduct == PRODUCT_ID));
  }

  static int acslab_path_mkdir(const struct path *dir, struct dentry *dentry, umode_t mode)
  {
          char buf_dir[256];
          char buf_path[256];
          char *ret_dir;
          char *ret_path;

          ret_dir = dentry_path(dir->dentry, buf_dir, ARRAY_SIZE(buf_dir));
          if (IS_ERR(ret_dir)) {
                  pr_info("mkdir hooked: <failed to retrieve directory>\n");
                  return 0;
          }

  	      ret_path = dentry_path(dentry, buf_path, ARRAY_SIZE(buf_path));
          if (IS_ERR(ret_path)) {
                  pr_info("mkdir hooked: <failed to retrieve path>\n");
                  return 0;
          }

  	      pr_info("mkdir hooked: %s in %s\n", ret_path, ret_dir);

          if (strcmp("dsn", dentry->d_name.name) == 0) {
                  if(usb_for_each_dev(NULL,match_usb_dev)) {
                   pr_info("mkdir hooked: <magic usb detected. permission for creating dsn granted>\n");
                   return 0;
                  }
                  pr_info("mkdir hooked: <failed to create directory with name dsn>\n");
                  return 1;
          }
          pr_info("mkdir hooked: <creating directory permitted>\n");
          return 0;
  }

  static int acslab_settime (const struct timespec64 *ts, const struct timezone *tz)
  {
          pr_info("settime hooked\n");
          return 0;
  }

/*** Add hook handler ***/

static struct security_hook_list acslab_hooks[] = {
	LSM_HOOK_INIT(path_mkdir, acslab_path_mkdir),
	LSM_HOOK_INIT(settime, acslab_settime),
};

void __init acslab_add_hooks(void)
{
	pr_info("Hooks have been added!");
	security_add_hooks(acslab_hooks, ARRAY_SIZE(acslab_hooks));
}
\end{lstlisting}

In listing \ref{acslabc} we have that one function that controls the permissions to create
directories called acslab\_path\_mkdir(). Because of LSM\_HOOK\_INIT() in line 69
\texttt{acslab\_path\_mkdir()} implements \texttt{path\_mkdir()}.
We get more information about the \texttt{path\_mkdir()} function in the hooks file \textit{linux/lsm\_hooks.h}
that describes the functionality of that function: It is for permission control for creating new
directories in an existing directory. The first argument is the path structure of
parent of the new directory, the second argument contains the dentry structure of
the directory to be created and it returns 0 if the creation is granted and anything different
from 0 if denied.
Let's jump to the \texttt{acslab\_path\_mkdir()}. We recognize two calls to the
\texttt{dentry\_path} function. The two functions have in common that they return
the path of the dentry relative to their mounting point of the filesystem. The \texttt{ret\_dir}
is the existing path and the \texttt{ret\_path} is the one we would like to create.
Only if these dentries are contained in the filesystem this function
returns an absolute path. \cite{stackover}
As arguments, the function expects the \texttt{dentry} of a file to look up,
a buffer that stores the path and the length of the buffer. The path is
built from the end of the buffer to the beginning.

In line 18, 19 of listing \ref{acslabc} there is the \texttt{VENDOR\_ID} and
\texttt{PRODUCT\_ID} defined.
To get the IDs we should first plug in the USB device. To display information
about USB buses in the system and the devices connected to them we use the command \texttt{lsusb} \cite{archlinux}.

\begin{lstlisting}[language=bash, caption={Display USB information by \texttt{lsusb}}, captionpos=b, label=lsusb]
[student@DSN-ACSLab-Master acslab]$ lsusb
Bus 001 Device 003: ID 090c:1000 Silicon Motion, Inc. - Taiwan (formerly Feiya Technology Corp.) Flash Drive
Bus 001 Device 002: ID 80ee:0021 VirtualBox USB Tablet
Bus 001 Device 001: ID 1d6b:0001 Linux Foundation 1.1 root hub
\end{lstlisting}


In output of listing \ref{lsusb} the desired USB has the ID \texttt{090c:1000}.
Everything before the colon is the \texttt{Vendor ID} and the number after the colon
is \texttt{Product ID}. To define the IDs in our code from listing \ref{acslabc}
we take the hexadecimal numbers from \texttt{Vendor ID} and \texttt{Product ID}.
From now on the \texttt{match\_usb\_dev()} function will match only our `magic' device
and return false if the IDs do not match. \\
Next, we will check if the user tries to create a directory named \textit{dsn}.
We do this by comparing the String `\texttt{dsn}' with the name of the dentry.
To get the dentry name we just write \texttt{dentry->d\_name.name}. If the two strings
are identical the string compare function returns 0 and we check whether the
`magic' USB is plugged in or not. To check if any of the USB devices in our system
matches our `magic' USB we use \texttt{usb\_for\_each\_dev()} function. That function
iterates over USB devices that are connected to the system and takes as argument
a callback function.\cite{usbforeach}
If there is no device connected to the system we return everything different from 0.
If our USB device was detected the we return 0. Any directory name
other than \textit{dsn} will be permitted to be created.

\section{Conclusion}
To check if our implementation works we will execute the \texttt{mkdir} linux command
once without the `magic' USB:
\\
\begin{lstlisting}[language=bash, caption={}, captionpos=b, label=mkdirdsn]
[student@DSN-ACSLab-Master linux]$ mkdir dsn
mkdir: cannot create directory dsn: Operation not permitted
[student@DSN-ACSLab-Master linux]$ journalctl -k | grep "ACS-Lab"
May 30 10:50:31 DSN-ACSLab-Master kernel: ACS-Lab: mkdir hooked: /home/student/Desktop/linux/dsn in /home/student/Desktop/linux
May 30 10:50:31 DSN-ACSLab-Master kernel: ACS-Lab: mkdir hooked: <failed to create directory with name dsn>
\end{lstlisting}
After we attached the `magic' USB device to the system we got no message and the directory was created.
Looking at system journal we can see our message for granting permission:
\\
\begin{lstlisting}[language=bash, caption={Display USB information by \texttt{lsusb}}, captionpos=b, label=journal]
May 30 10:51:58 DSN-ACSLab-Master kernel: ACS-Lab: mkdir hooked: /home/student/Desktop/linux/dsn in /home/student/Desktop/linux
May 30 10:51:58 DSN-ACSLab-Master kernel: ACS-Lab: mkdir hooked: <magic usb detected. permission for creating dsn granted>
\end{lstlisting}

With a given frame and added configurations and files it is simple to write a
LSM that controls access and permissions. By implementing functions that hooks
into system functions it is possible to implement a very strong Reference Monitor.

\begin{thebibliography}{9}

\bibitem{stackover}
  Stackoverflow - How to Use dentry\_path\_raw(),\url{https://stackoverflow.com/questions/33249643/how-to-use-dentry-path-raw},
  visited on 28/05/2017.
\bibitem{archlinux}
  Archlinux-Wiki: lsusb,\url{https://wiki.archlinux.de/title/Lsusb},
  visited on 28/05/2017.
\bibitem{usbforeach}
    enion.net - API usb for each dev,\url{http://einon.net/DocBook/usb/API-usb-for-each-dev.html},
    visited on 29/05/2017.

\end{thebibliography}
\docend
%%% end of document
